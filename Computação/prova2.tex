\documentclass[11pt]{article}
\usepackage[utf8]{inputenc}
\usepackage[a4paper, top=3cm, right=2cm, left=3cm, margin=4cm]{geometry}
\usepackage[brazilian]{babel}
\usepackage{graphicx}
\usepackage{indentfirst}
\usepackage{float}
\usepackage{amsmath}
\newcommand{\mat}[1]{\mbox{\boldmath{$#1$}}}

\title{Avaliação 2 - Introdução à Computação}
\author{Aluno(a): Antônio Arthur Silva de Lima}
\date{Semestre de 2021.1}

\begin{document}

\maketitle

\begin{abstract}
    Gosto de escutar rock clássico, bandas como Pink Floyd ou The Beatles são minhas favoritas, pois têm músicas calmas e relaxantes. Também gosto de alguns thrash metals, como Metallica ou Three Days Grace. Também leio muitos livros, principalmente livros de ficção-científica e fantasia.
\end{abstract}

\section{Equações, etiquetas e imagem}

\noindent Um pequeno desenvolvimento: 

\begin{eqnarray}
(a + b)^3 &=& \label{ref:polinomio1}
(a + b)(a + b)^2 \\ 
&=& (a + b)(a^2 + 2ab + b^2) \\
&=& a(a^2 + 2ab + b^2) + b(a^2 + 2ab + b^2) \\
&=& (a^3 + 2a^2b + ab^2) + (a^2b + 2ab^2 + b^3) \\
&=& a^3 + b^3 + 2ab(a + b) + ab(a + b). \label{ref:polinomio5}
\end{eqnarray}


Aplicando a identidade desenvolvida nas Equações (\ref{ref:polinomio1})-(\ref{ref:polinomio5}), temos \\

\noindent $(2+5)^3 = 2^3 + 5^3 + 2 \cdot 2 \cdot 5 \cdot (2+5) + 2 \cdot 5 \cdot (2+5) = 8 + 125 + 140 + 70 = 343.$ \\
    
A Figura \ref{fig:minastirith} mostra Minas Tirith: \\\\

\begin{figure}[H]
    \centering
    \includegraphics[width=5cm]{wallhaven-0jed9p_1920x1080.png}
    \caption{Minas Tirith, cidade capital do Reino de Gondor e o último refúgio da humanidade contra os poderes de Sauron (O Senhor do Escuro) dentro do universo criado por J.R.R. Tolkien.}
    \label{fig:minastirith}
\end{figure}

\end{document}
