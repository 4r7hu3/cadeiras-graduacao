\documentclass{article}
\usepackage[utf8]{inputenc}
\usepackage[portuguese]{babel}
\usepackage{indentfirst}
\usepackage{amsmath}
\usepackage{graphicx}
\usepackage{float}

\title{Meu Querido e Primeiro Projeto}
\author{Autor desconhecido}
%\date{July 2021}

\begin{document}

\maketitle

\begin{abstract}
    Este texto mostra alguns elementos básicos do \LaTeX. É lindo.
\end{abstract}

\section{Introdução}
% Aqui eu defino a notacao e os principais conceitos

Este é o \texttt{primeiro parágrafo}. Eu posso marcar o início de um novo parágrafo deixando uma linha em branco, conforme a seguir.\\

Só que isso não dá um grande espaçamento entre os dois parágrafos. Opcio\-nalmen\-te, eu posso colocar duas barras invertidas ao final do parágrafo, para ter um espaçamento maior antes de iniciar o próximo. Observe.\\

Agora, eu mostro como incluir conteúdo matemático dentro de um texto. Basta usar o símbolo de cifrão para marcar o início e o fim de cada \textbf{expressão matemática} dentro do texto. Chamamos isso de ``modo matemático''. Veja: a equação $a^2 = \mathbf{b^2} + c^2$ é linda.\\

Note como eu fiz para usar aspas acima: não foi usando o caractere de aspas duplas do teclado. Esta é a forma correta de usar aspas no \LaTeX. \textbf{Não use o caractere aspas duplas do teclado.} (Acabei de usar negrito para destacar a frase anterior.)\\

A seguir, eu mostro como fazer uma enumeração. O número é atribuído automaticamente. Quando tenho uma enumeração dentro de outra, a segunda assume uma numeração diferente. Experimente fazer um terceiro nível de enumeração, para ver o que acontece com a terceira lista.\\

Antes mostrar a enumeração, eu optei por quebrar a página com o comando \texttt{newpage}, para que a lista não ficasse dividida em duas páginas.\\

\newpage

% Exercicio 1: Faca mais uma enumeracao, abaixo da subopcao 2.
\begin{enumerate}
    \item Opção 1;
    \begin{enumerate}
        \item Subopção 1;
        \item Subopção 2.
        \begin{enumerate}
            \item Subsubopção (A);
            \item Subsubopção (B);
            \item Subsubopção (C).
        \end{enumerate}
    \end{enumerate}
    \item Opção 2;
    \item Opção 3.
\end{enumerate}

\begin{itemize}
    \item Item 1;
    \begin{itemize}
        \item Subitem 1;
        \begin{enumerate}
            \item Subsubopção (A);
            \item Subsubopção (B);
            \item Subsubopção (C).
        \end{enumerate}
        \item Subitem 2.
    \end{itemize}
    \item Item 2.
\end{itemize}

\section{Equações}

A seguir, mostro uma segunda maneira de entrar no modo matemático. Desta vez, o conteúdo matemático não está mais dentro do parágrafo, e sim isolado do restante do texto. Veja:
\[
 \sum_{i=1}^{n} \sqrt{\frac{ 1 }{ 2^i }} + \int_{\sqrt{\frac{ 1 }{ 2^i }}}^{b} \sqrt{i}
\]

\[
1,\ldots,n, \quad A_1 \times A_2 \times \ddots \times A_n, \quad \alpha \cdot f(x)
\]

Agora, eu vou tentar colocar o mesmo somatório acima dentro do parágrafo, misturado com o texto. Ou seja, vou colocar a expressão $\sum_{i=1}^{n} \frac{1}{2^i}$ como parte de uma frase. Veja como ela ficou compactada, no intuito de manter o espaçamento vertical padrão entre as linhas. Compare o somatório neste parágrafo com aquele que aparece acima. Eles possuem o mesmo conteúdo matemático, mas são formatados de maneiras distintas.\\

Esse recurso é muito conveniente, às vezes, mas eu posso querer mostrar o somatório naquele formato mais expandido, que ocupa mais espaço vertical. Isso é possível com o comando \texttt{displaystyle}. Veja que agora a expressão $\displaystyle \sum_{i=1}^{n} \frac{1}{2^i}$ ocupa mais espaço verticalmente. Consequentemente, o espaçamento vertical em torno da linha que contém o somatório fica diferente daquele das demais linhas no parágrafo.\\

Agora, tente fazer os dois exercícios\footnote{Acalme-se. Todos os exercícios são opcionais. Não precisa fazer agora, nem enviar a resposta para o professor.} que estão no código-fonte \LaTeX utilizado para gerar este documento (disponível no SIGAA).

\begin{equation}
    x + y = z.
\end{equation}

\begin{equation}
\int_{a}^{b} f(x)dx \label{eqdaintegralab}
\end{equation}

Conforme a equação (\ref{eqdaintegralab}), podemos ver que...

\subsection{O ambiente \texttt{eqnarray}}

Permite escrever sequências de equações, sem precisar criar múltiplos ambientes do tipo \texttt{equation}.\\

A equação (\ref{binomio}) mostra a identidade desejada.

\begin{eqnarray}
(a+b)^2 &=& (a+b)(a+b)\nonumber\\
&=& a^2 + ab + ba + b^2\nonumber\\
&=& a^2 + 2ab + b^2.\label{eq:binomio}
\end{eqnarray}

Conforme vimos na equação (\ref{eq:binomio}), ...

\section{Figuras}

Vamos ver como inserir figuras. Por exemplo, veja a figura \ref{fig:my_label}.

\begin{figure}[H]
    \centering
    \includegraphics[scale=0.3]{bonzao1.png}
    \caption{Um exemplo de bonzismo. Vai já cair.}
    \label{fig:my_label}
\end{figure}



\end{document}
