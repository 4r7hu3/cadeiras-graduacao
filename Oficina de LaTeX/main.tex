%Pacotes
\documentclass[12pt]{article} %Define o tipo de documento. Ex: Beamer (slide) 
\usepackage[utf8]{inputenc} % Compreender a acentuação feita direto pelo teclado. 
\usepackage[brazilian]{babel} % Idioma
\usepackage[a4paper,top=3cm,right=2cm,left=3cm,margin=2cm]{geometry} %Margens 
\usepackage{setspace} % Espaçamento entre linhas
\setstretch{1.5} % Espaçamento entre linhas
\usepackage{graphicx} % Permite a inserção de figuras ou gráficos no texto
\usepackage{indentfirst} % Criar parágrafos \indent \noindent
\usepackage{float} % Fixar as imagens no locais em que foram colocadas
\usepackage{amsmath} % Alinha as equações
\usepackage{listings} % Pacote para fazer itens 
\usepackage{xcolor} %Pacote de cores
\newcommand{\mat}[1]{\mbox{\boldmath{$#1$}}} 


\usepackage[utf8]{inputenc}

\title{Minicurso Latex}
\author{petestatisticaufc }
\date{Outubro 2021}

\begin{document}
%%%%%%%%%%%%%%%%% CAPA %%%%%%%%%%%%%%%%% 
\thispagestyle{empty} % Retirar o número da página
\begin{center}
  \begin{figure} [h]
      \centering
      \includegraphics[scale=0.1]{WhatsApp Image 2020-09-21 at 3.14.21 PM.jpeg}
    
      \label{fig:my_label}
  \end{figure}
  
UNIVERSIDADE FEDERAL DO CEARÁ \\ % Pular linha
CENTRO DE CIÊNCIAS \\
DEPARTAMENTO DE ESTATÍSTICA E MATEMÁTICA APLICADA\\

\vspace{2cm}

PET Estatística UFC\\

\vspace{4cm}

\textbf{TÍTULO DO TRABALHO} \\

\vspace{8cm}

\textbf{FORTALEZA}\\
\textbf{Outubro/2021}
    
\end{center}

%%%%%%%%%%%%%% SUMÁRIO %%%%%%%%%%% 
\newpage % Nova página
\begin{center}
\tableofcontents %Sumário

\newpage
\listoftables % Lista de tabelas
\listoffigures % Listas de Figuras
%%%%%%%%%%%%%%%%%%%%%%%%%%%%%%%%%% 

\end{center}

\newpage
% Criar uma seção
\section{Introdução}
\indent O \textcolor{red}{TeX} é um sistema de tipografia desenhado e escrito principalmente por Donald Knuth[1] e lançado em 1978. Junto com a linguagem para a descrição de fontes Metafont e a família de tipos de letras Computer Modern, o TeX foi projetado tendo em vista dois objetivos principais: permitir que qualquer pessoa produza livros de alta qualidade com o mínimo esforço e fornecer um sistema que produza exatamente os mesmos resultados em todos os computadores em qualquer momento. O TeX é um software livre, o que o torna mais acessível a um número maior de pessoas. \\ %Pula uma linha
\indent O TeX é uma alternativa conhecida para se digitar fórmulas matemáticas complexas, apontada como um dos sistemas de tipografia mais sofisticados do mundo. O TeX é popular na academia, especialmente em matemática, ciências da computação, economia, engenharia, física, estatística e psicologia quantitativa.\\

\section{Estrutura de texto}
\subsection{Tamanho} % Criar uma subseção
{\singlespacing % Colocar um espaçamento simples

\noindent \tiny{Letra super reduzida}\\
\scriptsize{\textbf{Letra bastante pequena}}\\
\footnotesize{Letra muito pequena}\\
\small{Letra pequena}\\
\normalsize{Letra normal} \\
\large{Letra grande}\\
\Large{Letra maior}\\
\LARGE{Letra muito grande}\\
\huge{Letra bastante grande}\\
\Huge{Letra super grande}\\
}

\newpage
\subsection{Fontes}
\noindent
\textbf{Negrito}\\
\textit{Itálico} \\
\texttt{Máquina de escrever}

\subsection{Alinhamento}
Usando o ambiente \textbf{center} o texto poderá ser centralizado, com o ambiente \textbf{flushleft} o texto é alinhado à esquerda e com o ambiente \textbf{flushright} o texto é alinhado à direita.

\begin{center}
    Centralizado
\end{center}

\begin{flushleft}
alinhado à esquerda
\end{flushleft}

\begin{flushright}
alinhado à direita
\end{flushright}

Exemplos\\

\begin{center}
   \textbf{\huge{Estatística}}\\
   \LARGE{O melhor curso!}\\
\end{center}

\begin{flushright}
    Venha fazer parte do PET Estatística! 
\end{flushright}

\begin{flushleft}
    \textit{Universidade Federal do Ceará - UFC}
\end{flushleft}

\section{Tabelas}
\begin{tabular}{r|c} % r - alinhado à direita, l - alinhado a esquerda e c - centralizado.
   
   \hline
   
   e  & b\\
   \hline
   c  & d\\
   \hline
   d & f \\
   \hline
   4 & 5\\
   
   \hline
   
\end{tabular}\\
Podemos observar na Tabela \ref{medidasresumo} % Referenciar Tabela

\begin{table}[H]
\caption{Medidas resumo}% Título da tabela 
\vspace{0.3cm}
\centering
\begin{tabular}{cclrrrr} % r - alinhado à direita, l - alinhado a esquerda e c - centralizado.
  \hline % Linha horizontal | \vline - linha vertical
 Dias & Média & DP & CV(\%) & Mínimo & Mediana & Máximo \\ 
  \hline
  

  1 & 4,5 & 2,0 & 45,0 & 0,0 & 4,4 &  9,1  \\ 
  2 & 6,1 & 2,9 & 50,6 & 0,0 & 5,8 & 11,4 \\ 
  3 & 5,8 & 3,9 & 65,8 & 0,0 & 5,9 & 14,6 \\ 
   \hline
\end{tabular}
\label{medidasresumo} % Nome da tabela para referenciar
\end{table}
\section{Ambiente Matemáticos}
Pode se introzir no ambiente matemático da forma \( e \),entre $ e $ ou entre \begin{math} e \end{math}.\\

\( 2+2 \)\\

\begin{math}
2+2
\end{math}\\

\begin{math}
\sum_{i=1}^2
\end{math}
\subsection{Símbolos matemáticos}
letras gregas  $\alpha, \beta,\gamma$.A raiz quadrada $\sqrt{4}$. Vetores: $A\longleftarrow B$
\subsection{Equações enumeradas e não-enumeradas}
Podemos adicionar enumeradas.

\begin{equation}
    \frac{n!}{2n!}
    \label{fração}
\end{equation}
A fração \ref{fração} foi enumerada 
Podemos adicionar não-enumeradas

\begin{equation*}
    \frac{1}{n!}
    \label{fraçãonãoenumeradas}
\end{equation*}
A fração  foi não-enumerada.
\subsection{Equações alinhadas}
Podemos alinhar equações.
\begin{align}
    C_{5,2} &= \frac{5!}{2!(5-2)!}\\
            &=  \frac{5!}{2!3!}
\end{align}
\begin{align*}
    &=\frac{6\cdot5\cdot4\cdot3\cdot2\cdot1}{2!3!}
\end{align*}
\subsection{Matriz}
\begin{equation}
\mat{X} =
       \begin{bmatrix}
       x_{11} & \hdots & x_{1z}\\
       \vdots & \ddots & \vdots\\
       x_{n1} & \hdots & x_{nz}
       \end{bmatrix}
\end{equation}
Exemplo:
\begin{equation*}
\mat{y} =
       \begin{bmatrix}
       12 & 23 & \hdots & 45 & 56\\
       23 & 90 & \hdots & 87 & 57\\
       \end{bmatrix}
\end{equation*}
% Aumentar o tamanho: {} , [] e ()
\begin{equation}
    f(x) =  \left \{ \frac{1}{x^2} \right \}\\
\end{equation}
%item
\section{Item}

\begin{item}
 \item Análise multivariada 
 \item Análise multivariada
\end{item}

\begin{itemize}
\item Análise combinatória
\end{itemize}
\begin{itemize}
\item Cálculo 1
\end{itemize}

\begin{itemize}
\item[1] Análise combinatória
\item[2] Cálculo 1
\end{itemize}

\begin{lstlisting}
print(ola mundo)
[1] ola mundo
\end{lstlisting}

\newpage
\section{Figuras}
Na figura \ref{fig:my_label} é apresentado um histograma.
\begin{figure}[H]
    \centering
    \includegraphics[scale=0.5]{histograma (1).png}
    \caption{Caption}
    \label{histograma}
\end{figure}
\begin{figure}[H]
    \centering
    \includegraphics[scale=0.2]{histograma (1).png}
    \caption{Caption}
    \label{fig:my_label}
\end{figure}











\end{document}